%%%%%%%%%%%%%%%%%%%%%%%%%%%%%%%%%%%%%%%%%%%%%%%%%%%%%%%%
% Este é um documento que servirá de modelo para
% os relatórios feitos na disciplina Circuitos Digitais
% 2016-2
%%%%%%%%%%%%%%%%%%%%%%%%%%%%%%%%%%%%%%%%%%%%%%%%%%%%%%%%%

\documentclass[12pt]{article}

\usepackage{sbc-template}
\usepackage[brazil,american]{babel}
\usepackage[utf8]{inputenc}

\usepackage{graphicx}
\usepackage{url}
\usepackage{float}
\usepackage{listings}
\usepackage{color}
\usepackage{todonotes}
\usepackage{algorithmic}
\usepackage{algorithm}
\usepackage{hyperref}

\sloppy

\title{Projeto Final\\ 
	Dados abertos do Bolsa Família}

\author{Dayanne Fernandes da Cunha, 13/0107191\\
	 Christian Costa Werner ,  14/0134573
}

\address{Dep. Ciência da Computação -- Universidade de Brasília (UnB)\\
	Banco de Dados - Turma A
	\email{dayannefernandesc@gmail.com, ccwerner96@gmail.com}
}

\begin{document} 
	\maketitle
	
	\selectlanguage{american}
	\begin{abstract}
		This report corresponds to the step-by-step project of creating a database using open data from the federal government (transparency portal). We will answer some issues about \textit{Bolsa Familia} theme.
	\end{abstract}
	\selectlanguage{brazil}     
	
	\begin{resumo} 
		Este relatório corresponde ao passo a passo do projeto de criação de um banco de dados utilizando dados públicos do governo federal (portal da transparência). Iremos sanar algumas questões sobre o tema \textit{Bolsa Família}.
	\end{resumo}
	
	\tableofcontents
	\newpage 
	
	\section{Introdução}
	\label{sec:intro}
	
	\section{DER}
	\label{sec:der}
	
	\section{MR} 
	\label{sec:mr}

	\section{Formas Normais}
	\label{sec:fnormais}
	
	\subsection{1F}
	\label{sec:1f}
	
	\subsection{2F}
	\label{sec:2f}
	
	\subsection{3F}
	\label{sec:3f}
	
	\section{Script geral}
	\label{sec:scriptg}
	
	\section{ETL}
	\label{sec:etl}
	
	\section{Camadas}
	\label{sec:camadas}
	
	\subsection{Persistência}
	\label{sec:pers}
	
	\subsection{GUI}
	\label{sec:gui}
	
	\section{Consultas} 
	\label{sec:consultas}
	
	\section{Views}
	\label{sec:views}
	
	\section{Procedure}
	\label{sec:procedure}
	
	\section{Trigger}
	\label{sec:trigger}
	
	\section{Solução das Perguntas}
	\label{sec:analisep}
	
	\section{Análise dos Resultados}
	\label{sec:resultados}
	
	\section{Conclusão}
	\label{sec:conclusao}
	
%\bibliographystyle{sbc}
%\bibliography{relatorio} 

%\newpage 
	
\end{document}
